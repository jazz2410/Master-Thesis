\chapter{Stand der Technik}

Wie bereits erwähnt, soll die Parametrisierung des in dieser Arbeit entwickelten Black-Box-Modells mit Methoden des maschinellen Lernens stattfinden. Da der Begriff des maschinellen Lernens ein weites Spektrum an theoretischen Grundlagen und praktischen Anwendungen umfasst, enthält der nächste Abschnitt grundlegende Erläuterungen zu diesem Begriff.


\section{Maschinelles Lernen}
\label{cha:machine_learning}
Der Begriff des maschinellen Lernens ist ein Oberbegriff, welcher eine Reihe von Methoden umfasst, welche den Zweck haben, Modelle auf Basis von gesammelten bzw. gemessenen Daten zu generieren. Je umfangreicher dabei die Datengrundlage ist, desto besser ist das Modell in der Lage, das Verhalten des realen Systems abzubilden.  Darüber hinaus ist das erzeugte Modell fähig, das Systemverhalten über die verwendete Datengrundlage hinweg zu verallgemeinern. Damit ist die Anwendung des Modells auf bisher unbekannte Datensätze derselben Art möglich, um daraus z.B. Vorhersagen über das zukünftige Systemverhalten zu machen.  Der entscheidende Unterschied zu klassischen Systemidentifikationsmethoden ist, dass die Parametrisierung des Modells automatisiert durch Algorithmen und nicht durch explizites Festlegen stattfindet. \cite{Duriez.2017} \\

Gemeinhin wird in der Literatur die Parametrisierung bzw. Adaptierung der Modelle durch Daten als \textit{Training} bezeichnet. Nach \cite{Dobel.2018} gibt es verschiedenen Lernverfahren mit folgenden Anwendungsgebieten:

\begin{itemize}
	\item \textit{Unüberwachtes Lernen:} Typische Anwendungsgebiete sind die Dimensionsreduktion (z.B. bei der Merkmalsextraktion), Generative Netzwerke (z.B. zur Musik- oder Bildgenerierung) und das Clustering.
	\item \textit{Semi-überwachtes Lernen:} Anwendungsfälle sind die Text-Klassifikation oder die Spurverfolgung auf Straßen.
	\item \textit{Bestärkendes Lernen:} Anwendungsfälle sind das Aneignen von Fähigkeiten oder Routinen bei Robotern oder die Roboternavigation.
	\item \textit{Überwachtes Lernen:} Anwendungsfälle sind die Regression (z.B. Wettervorhersagen, Lebensdauerabschätzung) und Klassifikation (z.B. Bildklassifikation, Fehlererkennung). \cite{Dobel.2018}
\end{itemize} 

Für diese Arbeit stellt das überwachte Lernen das entscheidende Lernverfahren dar. Beim überwachten Lernen sind in den Trainingsdaten die Sollvorgaben in Form von gemessenen Ausgangsgrößen für die 3D-Servo-Presse (z.B. die Stößelhöhe) zu den korrespondierenden gemessenen Eingangsgrößen (Spindel- und Exzentervorschübe) vorhanden. Damit ist gewährleistet, dass das durch die Parametrisierung erzeugte Modell das Verhalten des realen Systems möglichst gut abbildet. \\
Wie bereits erwähnt, gibt es im Bereich des überwachten Lernens zwei Anwendungsgebiete, zum einen die Klassifikation, zum anderen die Regression. Die gängigen Methode des maschinellen Lernens in diesen zwei Anwendungsgebieten sind in Tabelle \ref{tab:machine_learning} aufgelistet.

\begin{longtable}{p{7.5cm} p{7.5cm}}
	\caption{Methoden des maschinellen Lernens im Bereich Klassifikation und Regression} \\ \toprule
	% Definition des Tabellenkopfes auf der ersten Seite
	\textbf{Klassifikation} & \textbf{Regression} \\
	\hline
	\endfirsthead % Erster Kopf zu Ende
	% Definition des Tabellenkopfes auf den folgenden Seiten
	\caption{Lange Tabelle mit Longtable Fortsetzung}\\
	1 Spalte & 2 Spalte\\
	\hline
	\endhead % Zweiter Kopf ist zu Ende
	\hline
	
	% Ab hier kommt der Inhalt der Tabelle
	Logistische Regression (Regressionsmodell) & Lineare Regression (Regressiosmodell)\\ \hline
	Iterative Dichotomiser (Entscheidungsbaum) & Klassifikations- und Regressionsbaum (Entscheidungsbaum) \\ \hline
	Random Forests (Entscheidungsbaum) & Random Forests (Entscheidungsbaum) \\ \hline
	Stützvektormaschine (Kernmethode) & \\ \hline
	Feed-forward Network (künstliche neuronale Netze) & Feed-forward Network (künstliche neuronale Netze) \\ \hline
	Lernen eines Bayesschen Netzes (Bayessche Modelle) & \\ 
	
	 \bottomrule
	\label{tab:machine_learning} 
\end{longtable}  

Die Parametrisierung des Black-Box-Modells für die 3D-Servo-Pressen stellt eine Regressionsaufgabe dar. Dafür erscheint der Einsatz von neuronalen Netzen zweckmäßig.  






