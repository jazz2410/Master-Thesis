\chapter{Einführung}

Die Planung und Einführung von Fertigungssystemen geht mit Unsicherheiten einher. Dies ist bedingt durch die limitierte Gültigkeit von Annahmen in Bezug auf zukünftige Ereignisse während der Auswahl- und Entwicklungsphase von Fertigungstechnologien und Werkzeugmaschinen \cite{Groche.2010}. Nach \cite{Gerwin.1993} gibt es vier Arten von Unsicherheiten: die Marktakzeptanz von bestimmten Produkten, die Länge der Produktlebensphasen, spezifische Produkteigenschaften und die aggregierte Produktnachfrage. Ein mögliches Lösungskonzept zur Bewältigung dieser Unsicherheiten besteht darin, die Flexibilität von Fertigungssystemen zu erhöhen. Daraus ergeben sich nach \cite{Son.1987} folgende Vorteile: durch die Erhöhung dieser können erstens eine höhere Anzahl an Produkten und Produktvariationen (Werkzeugflexibilität) in den Fertigungsprozess integriert werden, zweitens erhöht sich die Adaptionsfähigkeit des Fertigungsprozesses auf eine Veränderung der Produktpalette (Produktflexibilität), drittens erhöht sich bei flexiblen Fertigungsprozessen die Adaptionsfähigkeit auf Veränderungen des Prozesses, z.B. durch technologische Entwicklungen (Prozessflexibilität) und viertens kann durch solche Fertigungssysteme flexibler auf Nachfrageschwankungen reagiert werden (Nachfrageflexibilität). \\ 

Bisher kommen Umformmaschinen vor allem bei großen Stückzahlen und ausgewählten Umformmethoden oder kleinen Stückzahlen und vorher speziell festgelegten Werkzeugverfahrwegen zum Einsatz \cite{Groche.2010}. Dadurch ist die Adaptionsfähigkeit auf Nachfrageschwankungen sehr eingeschränkt \cite{Schmoeckel.1991}. Dagegen bietet bringt die Integration von Servomotoren in Pressen neue Möglichkeiten der Flexibilisierung mit \cite{Groche.2004}. Um dem Anspruch einer höheren Flexibilität für Pressen gerecht zu werden, entwickelte das Institut für Produktionstechnik und Umformmaschinen (PtU) die   neuartige 3D-Servo-Presse. Diese verfügt über drei Antriebssysteme. Diese erlauben es der 3D-Servo-Presse, zuzüglich zur translatorischen Stößelbewegung eine Verkippung orthogonal zur Translationsbewegung durchzuführen. Dadurch ergeben sich insgesamt drei Freiheitsgrade: eine translatorische Hubbewegung und zwei rotatorische Kippbewegungen. Dadurch ist die Herstellung neuartiger Produktgeometrien und das Einbringen bestimmter Materialeigenschaften in den umgeformten Produkten durch definierte Werkzeugbewegungen denkbar. Beispielsweise lässt sich durch den Einsatz der 3D-Servo-Presse die Anzahl der Prozessschritte bei der Herstellung von Bauteilen mit sehr hohen Umformgraden durch die gezielte Steuerung des Materialflusses reduzieren \cite{Sinz.2018}. Durch den Einsatz der 3D-Servo-Presse soll damit dem Anspruch nach höherer Flexibilisierung und der damit einhergehenden höheren Wirtschaftlichkeit gerecht werden. \\








