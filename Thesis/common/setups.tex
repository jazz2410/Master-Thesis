% Dieses File dient f�r wichtige Einstellungen. �nderungen sind zu vermeiden.
%

% =================================================================================
% Anpassung Absatzformat
% =================================================================================
% Der Abstand der Zeilen betr�gt das 1,25-fache des Standard-Abstands von
% LaTeX. Da technische Arbeiten viele Formeln und Bilder enthalten, werden
% Abs�tze durch einen zus�tzlichen vertikalen Zwischenraum statt durch einen
% Einzug getrennt.
\linespread{1.25}
\setlength{\parindent}{0mm}
\setlength{\parskip}{1ex}

% =================================================================================

% =================================================================================
% Definitionen aus tudreport-Vorlage 
% =================================================================================

\newlength{\longtablewidth}
\setlength{\longtablewidth}{0.7\linewidth}
\addtolength{\longtablewidth}{-\marginparsep}
\addtolength{\longtablewidth}{-\marginparwidth}
% =================================================================================

% =================================================================================
% Farbige Deckseite, graue Balken auf allen anderen Seiten
% =================================================================================
\newif\ifOnlyColorFront
\OnlyColorFronttrue	% Balken nur auf Deckblatt farbig
%\OnlyColorFrontfalse		% alle Balken farbig

% =================================================================================
% Texte f�r den Titel und die R�ckseite des Titels vorgeben
% =================================================================================
\title{Entwicklung einer Regelung mit neuronalen Netzen f\"ur die 3D-Servo-Presse}
\subtitle{Tajinder Singh Dhaliwal}
\ifStdClassDraft
%
\else
    \subsubtitle{\SADATyp\ -- 31.01.2019  \\Betreuer: Florian Hoppe}
% \settitlepicture{}	% Bild f�r Deckblatt
% \printpicturesize

        \newcommand{\SADAProf}{Prof. Dr.-Ing. Dipl.-Wirtsch.-Ing. Peter Groche}
        \newcommand{\SADAinstitut}{Institut f�r Produktionstechnik und Umformmaschinen\\
                                   \SADAProf}
        \newcommand{\SADAwebsite}{www.ptu.tu-darmstadt.de}
        \newcommand{\SADAtel}{06151/16-3056}
        \newcommand{\SADAlogo}{\includegraphics[width=3cm]{./common/PtU_Logo.eps}}

    \setinstitutionlogo[height]{common/PtU_Logo.eps}

\fi
% Folgende Eintr�ge werden auf der R�ckseite der Titelseite gedruckt:
\uppertitleback{}
\lowertitleback{}
\dedication{}
% =================================================================================

% =================================================================================
% Befehle, die in scrreprt nicht exisiteren werden hier definiert, wenn diese
% Klasse verwendet wird.
% Auch die Seitenr�nder werden angepasst, so dass es grob wie mit tudreport
% aussieht.
% =================================================================================
\ifStdClassDraft
    % Diese Pakete nur extra einbinden, wenn NICHT tudreport als Basis.
	\usepackage{amssymb}
	\usepackage{geometry}
	\newcommand{\subsubtitle}[1]{}
	\newcommand{\settitlepicture}[1]{}
	\newcommand{\printpicturesize}{}
	\newcommand{\institution}[1]{}
    \pagestyle{headings}
	
	\author{\SADAAutor}	% scrreprt erwartet Autor
\fi
% =================================================================================


% =================================================================================
% Informationen (Meta-Daten) f�r pdf
% =================================================================================
\hypersetup{
	pdftitle = {\SADATitel},
	pdfsubject = {},
	pdfauthor = {\SADAAutor},
	pdfkeywords = {},
	pdfcreator = {},
	pdfproducer = {LaTeX with hyperref},
	pdfstartview = {Fit},
	pdfpagelayout = {SinglePage}
}
% =================================================================================

% =================================================================================
% Anpassung der Seitenr�nder
% =================================================================================
% f�r bessere Lesbarkeit nach der Bindung
\geometry{left=30mm, right=20mm, top=15mm, bottom=20mm}
% ================================================================================= 