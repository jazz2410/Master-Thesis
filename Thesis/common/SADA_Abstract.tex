\section*{Kurzfassung}
Diese Arbeit befasst sich mit der Entwicklung eines Verfahrens zur Beherrschung von Unsicherheiten in Drei-Punkt-Richtprozessen. Beim Richten ist sowohl fue Prozessgeschwindigkeit als auch Prozessgenauigkeit die genaue Praktion der kfederung des Bauteils von entscheidender Rolle. Dabei stellen oftmals schwankende Bauteileigenschaften eine Herausforderung dar,  die eine Regelstrategie benwird, die diese Schwankungen erkennen und kompensieren kann. Hierzu wird in dieser Arbeit ein Verfahren entwickelt, welches es erlicht, parallel zu einem Biegeprozess in Echtzeit alle relevanten Bauteilinformation aus der Kraft-Weg-Messung des Bauteils zu identifizieren und damit zu jedem Zeitpunkt die Rderung zu prktieren.\\
Anhand der Ergebnisse an einer Drei-Punkt-Richtmaschine wird gezeigt, dass mit diesem Verfahren auch ohne Kenntnis der Materialeigenschaften mit nur einem Richthub eine hohe Genauigkeit erzielt werden kann. Daer hinaus werden in Versuchen auch die Grenzen der Robustheit gegener Schwankungen in der Bauteilgeometrie getestet. Als Ausblick zu diesem Verfahren wird ein Lungsansatz geliefert, mit dem ein hes Man vertrauenswiger Information gewonnen werden kann und durch eine stochastische Modellierung der Unsicherheiten eine weitere Optimierunist. 

\textbf{Schter:} Drei-Punkt-Richten, Unsicherheit, Unwissen, Materialeigenschaft, Bauteileigenschaften, Robustheit


\selectlanguage{english}
\section*{Abstract} 

This thesis deals with the development of a method to control uncertainties in three-point straightening processes. Speed and accuracy in straightening processes are determined by its quality of springback prediction. Alternating material and part properties are a challenging task for springback prediction and require a control strategy which is able to detect and compensate those uncertainties. Therefore this thesis presents a method which is able to extract all essential information from the online force-displacement curvature during the straightening process and provides a real-time springback prediction.\\
Results from real processes on a three-point straightening machine have shown that this method is able to handle unknown uncertainties in material properties and achieve a high accuracy within one stroke. Additional results show the robustness of this method and its limits regarding uncertainties in part properties. A further solution is provided which gives an outlook on how to increase the amount of available, reliable information and therefore optimize the method with a stochastic uncertainty.

\textbf{Keywords:} three-point straightening, stochastic uncertainty, unknown uncertainty, material properties, part properties, robustness
\selectlanguage{ngerman} 