\usepackage{tikz}

\usetikzlibrary{arrows,calc}
\usetikzlibrary{positioning}
\usetikzlibrary{shapes}
\usetikzlibrary{backgrounds}
\usetikzlibrary{patterns}
\usetikzlibrary{fit}
\usetikzlibrary{decorations.markings}

\usepackage{pgfplots}


\newcommand{\TikZscale}{1}
\newcommand{\mm}{*\TikZscale mm}
% \input{d:/TikZ/TikZ_BSBnormal}

% TikZstyles f�r Blockschaltbilder
\tikzset{every picture/.style={auto, node distance=5\mm, >=stealth'}}

%% Bl�cke
	% Rechteckige Bl�cke
	\tikzstyle{blockS}     = [draw, rectangle, minimum height=15\mm, minimum width=30\mm, inner sep=3pt]
	\tikzstyle{NLblockS}   = [draw, rectangle, minimum height=15\mm, minimum width=30\mm, inner sep=3pt, double distance=4pt]	
	\tikzstyle{Trans}     = [draw, rectangle, minimum height=2\mm, minimum width=9\mm, inner sep=3pt]	
	
	\tikzstyle{block}      = [draw, rectangle, minimum height=10\mm, minimum width=10\mm, inner sep=3pt]
	\tikzstyle{NLblock}    = [draw, rectangle, minimum height=10\mm, minimum width=10\mm, inner sep=3pt, double distance=1.2pt]
	\tikzstyle{PICblock}   = [draw, rectangle, minimum height=10\mm, minimum width=10\mm, inner sep=2pt]
	\tikzstyle{NLPICblock} = [draw, rectangle, minimum height=10\mm, minimum width=10\mm, inner sep=3pt, double distance=1.2pt]
	\tikzstyle{noblock}		 = [rectangle, inner sep=-0.6pt]
	\tikzstyle{port}			=[draw, rectangle, minimum height=4\mm, minimum width=7\mm,rounded corners=2\mm]

	% Dreieckige Bl�cke
	\tikzstyle{Rgain}			 = [draw, isosceles triangle, inner sep=1pt, minimum width=13\mm, isosceles triangle apex angle=43]
	\tikzstyle{Lgain}			 = [draw, isosceles triangle, inner sep=1pt, minimum height=9\mm, isosceles triangle apex angle=60, shape border rotate=180]
	\tikzstyle{Ugain}			 = [draw, isosceles triangle, inner sep=1pt, minimum height=9\mm, isosceles triangle apex angle=60, shape border rotate=90]
	\tikzstyle{Dgain}			 = [draw, isosceles triangle, inner sep=1pt, minimum height=9\mm, isosceles triangle apex angle=60, shape border rotate=-90]

	% Runde Bl�cke
	\tikzstyle{sum}   	   = [draw, circle, inner sep=1pt, minimum size=3\mm]
	\tikzstyle{branch}		 = [draw, circle, inner sep=0pt, minimum size=1\mm, fill=black]


%% Verbindungselemente
	% Linien mit Pfeil
	\tikzstyle{bo}		=[dashed, thin]
	\tikzstyle{to}  		= [->, thick]
	\tikzstyle{toNL}		= [->, thick, shorten >=0.6pt]
	\tikzstyle{NLto}		= [->, thick, shorten <=0.6pt]
	\tikzstyle{NLtoNL}	= [->, thick, shorten <=0.6pt, shorten >=0.6pt]
	
	% Doppellinien mit Pfeil
	\tikzstyle{TO}  		= [semithick, double distance=2pt, shorten >=2mm, decoration={markings,mark=at position 1 with {\arrow[semithick]{open triangle 60}}}, postaction={decorate}]
	
	\tikzstyle{TONL} = [semithick, double distance=2pt, shorten >=2.2mm, decoration={markings,mark=at position 1 with {\arrow[semithick]{open triangle 60}}, transform={xshift=-0.7pt}}, postaction={decorate}]
	
	\tikzstyle{NLTO} = [semithick, double distance=2pt, shorten <=0.79pt, shorten >=2mm, decoration={markings,mark=at position 1 with {\arrow[semithick]{open triangle 60}}}, postaction={decorate}]
	
	\tikzstyle{NLTONL} = [semithick, double distance=2pt, shorten <=0.79pt, shorten >=2.2mm, decoration={markings,mark=at position 1 with {\arrow[semithick]{open triangle 60}}, transform={xshift=-0.7pt}}, postaction={decorate}]
	

	% Linien ohne Pfeil 
	\tikzstyle{line}       = [thick]
	\tikzstyle{lineNL}     = [thick, shorten >= 0.6pt]
	\tikzstyle{NLline}     = [thick, shorten <= 2pt]
	\tikzstyle{NLlineNL}	 = [thick, shorten <= 0.6pt, shorten >= 0.6pt]
	\tikzstyle{LineBS}	 = [line width=0.6mm]

	
	% Doppellinien ohne Pfeil
	\tikzstyle{LINE}       = [semithick, double distance=2pt]
	\tikzstyle{LINENL}     = [semithick, double distance=2pt, shorten >= 0.6pt]
	\tikzstyle{NLLINE}     = [semithick, double distance=2pt, shorten <= 0.6pt]
	\tikzstyle{NLLINENL}	 = [semithick, double distance=2pt, shorten <= 0.6pt, shorten >= 0.6pt]
	
	
	% Pins und Labels
	\tikzstyle{every pin edge}	= [<-, thick]
	\tikzstyle{every pin}	 			= [pin distance=5\mm]
	\tikzstyle{every label}			= [font=\small, label distance=-2pt]
	\tikzstyle{terminal}				= [coordinate]




\newcommand{\mywidth}{40mm}
\newcommand{\myheight}{25mm}